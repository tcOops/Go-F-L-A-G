\documentclass{article}
\usepackage[margin=1in]{geometry}
\usepackage[Glenn]{fncychap}
\usepackage{listings}
\usepackage{color}
\usepackage{verbatim}
\usepackage{ctex}
\usepackage{url}

\title{Hihocoder题目泛做解题报告}
\author{RejudgeX @ SJTU}

\definecolor{dkgreen}{rgb}{0,0.6,0}
\definecolor{gray}{rgb}{0.5,0.5,0.5}

\definecolor{mauve}{rgb}{0.58,0,0.82}

\lstset{frame=tb,
  language=c++,
  aboveskip=3mm,
  belowskip=3mm,
  showstringspaces=false,
  columns=flexible,
  basicstyle={\small\ttfamily},
  numbers=none,
  numberstyle=\tiny\color{gray},
  keywordstyle=\color{blue},
  commentstyle=\color{dkgreen},
  stringstyle=\color{mauve},
  breaklines=true,
  breakatwhitespace=true
  tabsize=3
}

\begin{document}
\begin{titlepage}
\maketitle
\thispagestyle{empty}
\pagebreak
\pagestyle{plain}
\tableofcontents
\end{titlepage}
\section{滚粗日记}

\subsection{1001 : A+B}
Default , Hello World :)
\lstinputlisting{Archive/1000.cpp}

\subsection{1014 : Trie树}
\begin{itemize}
\item 经典字典树(Trie)
\item Trie是一种基于字符串前缀保存的数据结构,将每个字符串以一棵树的形式保存,而相同的前缀会在同一颗树的路径上.
\item 关于Trie常见的操作有:Build, Update, Query. Update | Query的时候由于仅仅按照字符串的长度进行逐位操作,所以复杂度为$O(d)$, build的复杂度为$O(n*d)$.
\item Trie的扩展有基于Trie的贪心Xor(经典问题如:查找N个数最大的两个异或值, 查找区间最大的异或和等).
\end{itemize}
\lstinputlisting{Archive/1014.cpp}


\subsection{1015 : KMP算法}
\begin{itemize}
\item 经典KMP
\item KMP太经典了, 关于匹配串与模式串的匹配问题, 利用的是最长前缀与后缀相等来减少不必要的匹配,将匹配的复杂度从$O(n*m)$降到$O(n+m)$.
\item KMP的next数组可以做很多扩展,最常见的是求循环节。 注意理解next数组的本质,本质上next是多次迭代的,而每次迭代始终保持前缀与后缀相等,这个性质同样重要。
\end{itemize}
\lstinputlisting{Archive/1015.cpp}

\subsection{1032 : 最长回文子串}
\begin{itemize}
\item 经典Manacher求最长回文串
\item 字符串最长回文串最简单的就是$O(n^2)$的暴力匹配(枚举中心点), 当然也可以求正向串与逆向串的最长公共子序列,也是$O(n^2)$. 也可以通过枚举i点的前缀与后缀,基于后缀数组与高度数组,求后缀与前缀的最长公共前缀(LCP)来解决,但是复杂度与效率仍然不及Manacher.
\item Manacher的算法细节不表,说一下大概思路:在遍历字符串的过程中使用记录两个变量值$mx,idx$分别表示迭代到当前位置,最远的回文串能达到的位置以及其对应回文中心的下标。考虑对称性,迭代到i的时候,考虑i关于idx的对称点j,这样通过判断可以将j的回文长度加入到i的回文串中,这部分计算之前已经产生,所以节省了大量重复计算。最终比较一下所有位置的回文串长度即可。
\item 算法的复杂度为$O(n)$。有很多细节,比如预处理,位置比较等等,可以参考:\_{https://www.felix021.com/blog/read.php?2040}
\end{itemize}
\lstinputlisting{Archive/1032.cpp}

\subsection{1033 : 交错和}
\begin{itemize}
\item 数位DP
\item 数位DP是一类按位进行DP统计的问题。按位统计之后,发现有很多重复子问题,于是就可以愉快的DP了.
\item 数位DP的写法比较固定,一般都是用记忆化搜索的写法(参考我的代码. 注意需要考虑前导0,以及每次枚举的数字的范围(能不能到9).
\item 对于本题dp[len][sum]记录的是长度为len的数字序列交错和能达到sum的情况有多少种. 复杂度$O(d*k)$
\end{itemize}
\lstinputlisting{Archive/1033.cpp}

\subsection{1039 : 字符消除}
\begin{itemize}
\item 枚举、模拟
\item 考虑到字符串s的长度不超过100,完全可以暴力枚举解决。
\item 枚举可以替换的位置,将该位置分别填充为'A','B','C'进行消除,消除的过程简单模拟即可.复杂度$O(n^2)$
\end{itemize}
\lstinputlisting{Archive/1039.cpp}

\subsection{1040 : 矩形判断}
\begin{itemize}
\item 模拟题
\item 考虑围成矩形的条件:(1)四条边首尾相连,能形成一个闭合的环。 (2)这个环中的(1,3)(2,4)边互相平行.
\item dfs4条边所有的全排列,check是否满足上面两个条件. 时间复杂度$O(1)$
\end{itemize}
\lstinputlisting{Archive/1040.cpp}

\subsection{1041 : 国庆出游}
\begin{itemize}
\item DFS标记、BitSet
\item 题目的操作对象是一颗树, 对于树上的部分节点有先后遍历顺序要求,现在要访问所有节点一次,且树上的每条边都只访问两次(来回各一次
\item 从部分节点的遍历的顺序要求入手,只要能满足这部分节点的遍历顺序就一定找到解. 在树上遍历当然是dfs比较方便,dfs到当前点需要考虑当前需要的次序是否在这个点的子树下面,如果在就dfs下去,且标记过程中所有的边,如果全都找不到就说明无解。
\item 所以可以先dfs预处理一遍,找出节点u的子树包含的节点集合这样就能在O(1)时间进行判断. 考虑到n不大,直接用bitset搞定.
\end{itemize}
\lstinputlisting{Archive/1041.cpp}

\subsection{1042 : 跑马圈地}
\begin{itemize}
\item 大模拟
\item 首先这题的正解我并不知道是什么, 我AC的算法基于:最后圈出的一定是一个矩形。
\item 基于这一点,我分类大讨论了矩形与水塘的位置关系所有情况。 对于每种情况,枚举圈出的矩形的长宽,然后判断需要的周长是否不超过L,如果是就统计面积,进行比较.
\item 总之就是个大模拟辣 T\_T, 复杂度$O(n\*m\*(n+m))$?
\end{itemize}
\lstinputlisting{Archive/1042.cpp}

\subsection{1044 : 状态压缩·一}
\begin{itemize}
\item 状态压缩DP
\end{itemize}
\lstinputlisting{Archive/1044.cpp}

\subsection{1039 : 字符消除}
\begin{itemize}
\item 枚举、模拟
\item 考虑到字符串s的长度不超过100,完全可以暴力枚举解决。
\item 枚举可以替换的位置,将该位置分别填充为'A','B','C'进行消除,消除的过程简单模拟即可.复杂度$O(n^2)$
\end{itemize}
\lstinputlisting{Archive/1039.cpp}
\end{document}